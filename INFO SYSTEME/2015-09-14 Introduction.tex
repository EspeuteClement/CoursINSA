\documentclass[11pt]{article}

\usepackage{custom}
\usepackage{lipsum}
\usepackage{fancyhdr}
\usepackage{polyglossia}
\setmainlanguage{francais}
%mathieu.maranzana@insa-lyon.fr
\title{Système}
\author{{\sc Espeute} Clément}
\date{14 septembre 2015}


\begin{document}
\pagestyle{fancy}
\maketitle

\newpage
\section*{Equipe pédagogique système}
\subsection{Objectis généraux}

Moyenne UE >= 10, EC >= 7 (sinon pas de compensation)

Maitriser les concepts et principes fondamentaux des sytsèmes d'exploitation (architechture, organisation, interruption, fonctionnement ...)

Savoir définir, sépcifier et concevoir une architechture complexe (matérielle et logicielle) qui erspecte un certain nombre de contraintes et d'exigences

\subsection{Volume horaire}
3IF 22.5 h de cours, 6 séances (2h) de TD, 6 séances de TP (4h) total 58.5 h

\subsection{Projet longue durée}
Système et réseau 8 sécances

\subsection{Contenu}
Système d'exploitation Linux + shell

Makefile 1 - 2

Debugger (TD 3)

Multitâche linux (3 TD)


TP initiation linux + kde (EP-TP1)

TP initiation réseau IF (EP-TP2)

Mutitache Temps Partagé (PC-TP1\&2)

\subsection{Intéractions}
Architechture matériel / logiciel système

Ingénierie Projet (qualité et technique de tests, conduite de projets)

Méthode de développement de logiciels (Languages C / C++ ,Orienté Objet)

Réseaux

\subsection{Biblio}
Shell (Le système UNIX Inter editions)

Bash Référence Manual (v4.3) 

GNU make (v4.1)
\label{sub:}

% subsection  (end)
\label{sub:}

% subsection  (end)
\label{sub:}

% subsection  (end)
\end{document}