\documentclass[11pt]{article}

\usepackage{custom}
\usepackage{lipsum}
\usepackage{fancyhdr}
\usepackage{polyglossia}
\setmainlanguage{francais}

\title{Module Bases de Données Relationelles (BDR)}
\author{{\sc Espeute} Clément}
\date{14 septembre 2015}

\begin{document}
\pagestyle{fancy}
\maketitle
\newpage
\section{Introduction}
\subsection{Compétences}
Créer gérer une base de données, Interroger, Concevoir des applications, Intégration d'une base de données relationnelle.
\subsection{Contexte}
Gérer des données très volumineuses. Stocker, renouveler, mettre à jour les dionées -> Base de données permettant de les structurer, d'assurer leur intégrité, via des langages de haut niveau.

\emph{Examples de données p10 poly}

\subsection{Données dans les BDs}
Hypothése du monde clos (tout ce qui est dans la BDs est vrai, le reste est considéré faux)

\subsection{Systeme de gestion de Fichiers}
Offre une vue des données indépendante du support physique utilisé. Les données sont stockes dans des fichiers qui sont accéder par des applications. Problèmes : Redondance des données entre applications

\subsection{Systeme de gestion de base de données}
Plus de fichiers, les données sont stockées dans la base de donnée. Permet de répondre au besoin de plusieurs applications et plusieurs utilisateurs.

\section{SGBD}
Mets en relation les données. Manipulation simple avec un langage déclaratif.
Exemple {\tt SELECT ... FROM ... WHERE ...}

Gestion de la choérence (vérification des valeurs, des types, relations entre les autres données).

Permet de gérer les accès concurrents, en gérant les mise à jours conflictuelles.

\section{Applications}
Gestion, production, fichiers administratifs

\subsection{Intro}

Clé : un attribut qui permet d'identifier de manière unique, et il est toujours déclaré.

Clé primaire : Celle qui est choisie

Clé étrangère : Clé qui référence une autre table distante

\subsection{Types de donnés}
Numériques {\tt INT, SMALLINT, FLOAT, FLOAT(précision), NUMBER(précision, echelle)}

Chaînes de caractères {\tt CHAR(Longeur) (fixe), VARCHAR2(Longeur max) (variable) }

Temporelles {\tt DATE (yyy-mm-hh), HEURE (HH:MM:SS), TIMESTAMP, Sysdate}

Sequences Binaires

\subsection{Création de table}
%TODO Metre environement Verbatim
{\tt CREATE TABLE Licencie ( Nom varchar2(15),Prenom varchar2(15) ...)}

\subsubsection{Ajout de contrainte de clé primaire :}

{\tt CONSTRAINT nom\_de\_la\_contrainte PRIMARY KEY(nom\_de\_la\_clef)}

\subsubsection{Clé étrangères :}

{\tt CONSTRAINT nom\_de\_la\_contrainte FOREIGN KEY (nom\_de\_la\_clef) REFERENCES nomTable}

Permet de vérifier que le ta table référencé existe quand on crée un tuple

On peut uttiliser ON DELETE CASCADE ou ON DELETE SET NULL pour gérér les supréssions avec contraintes

\subsection{}
\end{document}