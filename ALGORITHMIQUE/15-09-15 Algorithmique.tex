\documentclass[11pt]{article}

\usepackage{custom}
\usepackage{lipsum}
\usepackage{fancyhdr}
\usepackage{polyglossia}
\usepackage{verbatim}
\setmainlanguage{francais}

\title{Algorithmique}
\author{{\sc Espeute} Clément}
\date{14 septembre 2015}

\begin{document}
\pagestyle{fancy}
\maketitle
\newpage

\section{Pseudo code}
Procédure : moyenne(a,b)
Entrée : entier a
			entier b
			
sortie reel résultat, correspondant à (a+b)

\begin{verbatim}
début
  | résultat <- (a+b)
  | résultat <- résultat/2
\end{verbatim}

Exemple 

\begin{verbatim}
Procédure racines(a,b,c,r1,r2)
 | Entrée   : reel a
 |            reel b
 |            reel c
 | Sortie   : reel r1
 |            reel r2
 | Précondition : b^2 - 4ac >= 0
\end{verbatim}

Instructions :

Expressions arithmétiques
\begin{verbatim}
	e1 <- (10 mod 3) + (5 div 2)
	e2 <- e1 * 2
	r1 <- 4.5 * 2.0
\end{verbatim}

Opérations de comparaison

Opérations logiques
\begin{verbatim}
	b1 <- (r1 > 4.2) ou (e1 = e2)
	b2 <- non(b1) et r1 <= 4.2
\end{verbatim}

Enchaînements

Séquentiel

Coût de l'algorithme : Nbre d'instruction effectuée pondéré du coup en mémoire de l'instruction

Enchainement alternatif

\begin{verbatim}
	si condition alors
	  |  --instruction
	sinon
	  |  -- autre instruction
\end{verbatim}

Repetitions

\begin{verbatim}
	tant que condition faire
	  | 
\end{verbatim}
\end{document}